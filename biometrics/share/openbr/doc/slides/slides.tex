\documentclass[12pt]{beamer}
\mode<presentation> {
  \usetheme{Madrid}
  \setbeamercovered{invisible}
  \setbeamertemplate{navigation symbols}{}
}

\usepackage{graphicx}
\usepackage[font={scriptsize}]{caption}

\title[OpenBR]{OpenBR -- Open Source Biometric Recognition}
\subtitle{and Beyond}
\author[J. Klontz]{Josh Klontz}
\institute[] { \small \emph{www.openbiometrics.org} }
\date{\today}

\begin{document}

\begin{frame}
\titlepage
\end{frame}

\section{Introduction}
\begin{frame}
\frametitle{Why Open Source?}
\pause
\begin{block}{Reproducible Research}
Support a common set of file formats and tools for algorithm design, development, and evaluation.
\end{block}
\pause
\begin{block}{Decrease Time to Market}
Provide a well-engineered and consistent framework for deploying new algorithms.
\end{block}
\pause
\begin{block}{Reduce Duplication}
Supply state-of-the-art baseline components for algorithm design.
\end{block}
\pause
\begin{block}{Improve Collaboration}
Help foster a community where collaboration takes place at the source code level.
\end{block}
\end{frame}

\begin{frame}
\frametitle{What's in it?}
\begin{block}{Off-the-shelf algorithms}
  \vspace{-6pt}
  \begin{columns}
  \column{0.48\textwidth}
    \begin{itemize}
    \item Face Recognition
    \item Gender Classification
    \end{itemize}
  \column{0.48\textwidth}
    \begin{itemize}
    \item Age Estimation
    \item Commercial Wrappers
    \end{itemize}
  \end{columns}
\end{block}
\pause
\begin{block}{Tools for algorithm evaluation}
  \begin{itemize}
  \item Standardized set of file formats
  \item Automatic plot generation
  \item Command line interface supporting common biometrics tasks
  \end{itemize}
\end{block}
\pause
\begin{block}{Software framework for algorithm development}
  \begin{itemize}
  \item C++ plugin API for implementing new algorithms
  \item Grammar for image processing
  \item Automatic testing, packaging and deployment
  \end{itemize}
\end{block}
\end{frame}

\newcommand{\graphicblock}[3]
{
  \begin{block}{#1}
    \begin{columns}
      \column{0.1\textwidth}
        \includegraphics[width=\textwidth]{#2}
      \column{0.9\textwidth}
        #3
    \end{columns}
  \end{block}
}

\begin{frame}[fragile]
\frametitle{Software Architecture}
\begin{columns}
\column{.48\textwidth}
  \graphicblock{Qt}{img/qt}{Cross-platform application and UI framework}
  \graphicblock{OpenCV}{img/opencv}{Image processing library}
  \graphicblock{Eigen}{img/eigen}{Linear algebra library}
  \graphicblock{CMake}{img/cmake}{Cross-platform build system}
\pause
\column{.48\textwidth}
  \graphicblock{br}{img/openbr}{Command line application for running algorithms and evaluating results.}
  \graphicblock{C API}{img/openbr}{High-level interface for other programming languages.}
  \graphicblock{C++ Plugin API}{img/openbr}{Core interface for using and developing algorithms.}
\end{columns}
\end{frame}

\begin{frame}
\frametitle{Supported Platforms}
\begin{columns}
\column{.3\textwidth}
  \begin{block}{Now}
  \centering
  \includegraphics[height=0.2\textheight]{img/windows} \\
  \includegraphics[height=0.2\textheight]{img/mac} \\
  \includegraphics[height=0.2\textheight]{img/linux}
  \end{block}
\pause
\column{.3\textwidth}
  \begin{block}{Soon}
  \centering
  \includegraphics[height=0.15\textheight]{img/android} \\
  \includegraphics[height=0.15\textheight]{img/ios} \\
  \includegraphics[height=0.15\textheight]{img/rpi} \\
  \includegraphics[height=0.15\textheight]{img/mpi}
  \end{block}
\pause
\column{.3\textwidth}
  \begin{block}{Future}
  \centering
  \includegraphics[height=0.2\textheight]{img/phi} \\
  \includegraphics[height=0.2\textheight]{img/nvidia} \\
  \includegraphics[height=0.2\textheight]{img/apu}
  \end{block}
\end{columns}
\end{frame}

\begin{frame}
\frametitle{Algorithm Evaluation}
\begin{figure}
\begin{columns}
  \column{0.33\textwidth}
    \includegraphics[width=\textwidth]{img/MEDS/MEDS_ROC}
  \column{0.33\textwidth}
    \includegraphics[width=\textwidth]{img/MEDS/MEDS_SD}
  \column{0.33\textwidth}
    \includegraphics[width=\textwidth]{img/MEDS/MEDS_BC}
\end{columns}
\caption{OpenBR vs COTS face recognition on \emph{MEDS} mugshot database.}
\end{figure}
\vspace{-20pt}
\pause
{\footnotesize
\hskip-10pt
\begin{tabular}{l | c | c | c | c | c}
 & \textbf{OpenBR} & \textbf{COTS-A} & \textbf{COTS-B} & \textbf{COTS-C} & \textbf{COTS-D} \\
\hline
\textbf{TAR @ FAR = 0.01} & 0.77 & 0.93 & \underline{0.96} & 0.86 & 0.80 \\
\textbf{Template Size (kB)} & \underline{0.75} & 2.8 & 5.0 & 36 & 74 \\
\textbf{Enrollment Speed} & \underline{10} & N/A & N/A & 1.3 & 1.2 \\
\textbf{Comparison Speed} & \underline{3,800,000} & N/A & 110,000 & 19,000 & 2,000 \\
 \end{tabular}
}
\end{frame}

\begin{frame}
\frametitle{FRVT 2012 (OpenBR = 'K')}
\centering
\includegraphics[height=0.85\textheight]{img/frvt2012a}
\end{frame}

\begin{frame}
\frametitle{FRVT 2012 (OpenBR = 'K')}
\includegraphics[width=\textwidth]{img/frvt2012b}
\end{frame}

\begin{frame}[fragile]
\frametitle{Algorithm Example: Face Recognition}
\begin{exampleblock}{}
\begin{verbatim}
$ br -algorithm FaceRecognition -compare me.jpg you.jpg
\end{verbatim}
\end{exampleblock}
\pause
\begin{block}{FaceRecognition}
FaceDetection!\textless FaceRegistration\textgreater !\textless FaceExtraction\textgreater + \textless FaceEmbedding\textgreater +\textless FaceQuantization\textgreater :UCharL1
\end{block}
\pause
\begin{block}{FaceDetection}
Open+Cvt(Gray)+Cascade(FrontalFace)
\end{block}
\pause
\begin{block}{FaceRegistration}
ASEFEyes+Affine(88,88,0.25,0.35)+FTE(DFFS)
\end{block}
\pause
\vspace{-9pt}
...
\vspace{-3pt}
\begin{block}{FaceEmbedding}
Dup(12)+RndSubspace(0.05,1)+LDA(0.98)+Cat+PCA(768)
\end{block}
\end{frame}

\begin{frame}
\frametitle{Live Coding}
\pause
\includegraphics[width=\textwidth]{img/livecoding1}
\vspace{12pt}
\includegraphics[width=\textwidth]{img/livecoding2}
\begin{block}{Inventing on Principle}
http://www.youtube.com/watch?v=PUv66718DII
\end{block}
\end{frame}

\begin{frame}
\frametitle{CPU Scaling}
\begin{figure}
\includegraphics[height=0.75\textheight]{img/cpuscaling}
\caption{http://www.extremetech.com/computing/116561-the-death-of-cpu-scaling-from-one-core-to-many-and-why-were-still-stuck}
\end{figure}
\end{frame}

\begin{frame}
\frametitle{Evolution of Hardware and Software}
\begin{figure}
\includegraphics[width=\textwidth]{img/amdscaling}
\caption{http://www.extremetech.com/computing/116561-the-death-of-cpu-scaling-from-one-core-to-many-and-why-were-still-stuck}
\end{figure}
\end{frame}

\begin{frame}
\frametitle{Hardware Realities}
\vspace{-12pt}
\begin{columns}
\column{0.33\textwidth}
\begin{figure}
\includegraphics[width=\textwidth]{img/i73930k}
\caption{i7 3930k}
\end{figure}
\column{0.33\textwidth}
\begin{figure}
\includegraphics[width=\textwidth]{img/gtx680}
\caption{GTX 680}
\end{figure}
\column{0.33\textwidth}
\vspace{9pt}
\begin{figure}
\includegraphics[width=\textwidth]{img/phi5110p}
\vspace{9pt}
\caption{Xeon Phi 5110p}
\end{figure}
\end{columns}
\pause
\begin{columns}
\column{0.33\textwidth}
\centering
\$570.00
\column{0.33\textwidth}
\centering
\$568.50
\column{0.33\textwidth}
\centering
\$2,649
\end{columns}
\pause
\begin{columns}
\column{0.33\textwidth}
\centering
\textbf{76.8 GFLOPS}
\column{0.33\textwidth}
\centering
\textbf{1665 GFLOPS}
\column{0.33\textwidth}
\centering
\textbf{1011 GFLOPS}
\end{columns}
\pause
\begin{alertblock}{Gotcha: Memory Bandwidth}
\begin{columns}
\column{0.3\textwidth}
\centering
\textbf{12.8 GFLOPS}
\column{0.3\textwidth}
\centering
\textbf{48.0 GFLOPS}
\column{0.3\textwidth}
\centering
\textbf{80 GFLOPS}
\end{columns}
\end{alertblock}
\pause
\begin{alertblock}{Gotcha: Code Duplication}
Need a separate code base for optimized performance on each device!
\end{alertblock}
\end{frame}

\begin{frame}
\frametitle{Requirements}
\begin{block}{What we want}
  \begin{itemize}
  \item Write once and run everywhere
  \item Automatically utilize all available hardware
  \item Run faster on future hardware
  \end{itemize}
\end{block}
\pause
\begin{block}{What we need}
  \begin{itemize}
  \item Virtual machine or just-in-time compiler
  \item Express computations using induction variables or ``kernels'': \\
           \center{\emph{void example\_kernel(int *a, int *b, int i) \{ a[i] += b[i]; \}}}
  \end{itemize}
\end{block}
\pause
\begin{block}{What we're proposing}
  \begin{itemize}
  \item LLVM IR and JIT compiler
  \item Designing for OpenCL 2.0 standard
  \end{itemize}
\end{block}
\end{frame}

\begin{frame}[fragile]
\frametitle{Goals}
\begin{block}{Perfectly Composable Image Processing Primitives}
A grammar for building algorithms from orthogonal primitive kernels with typeless semantics and optimized execution.
\end{block}
\pause
\begin{block}{When we say...}
\begin{verbatim}
Transform *lbpu2 = Transform::make("LBP(1)+U2");
\end{verbatim}
\end{block}
\pause
\begin{block}{...we mean}
Give me a pointer to a function that computes $LBP_{8,1}^{u2}$ on an image,
minimizes main memory transactions by combining kernels,
and is optimized for parallel execution on the hardware available.
\end{block}
\pause
\begin{exampleblock}{Take-Home Message}
$Compilation = Source\ Code + Available\ Hardware + First\ Image$
\end{exampleblock}
\end{frame}

\begin{frame}
\frametitle{The End}
\begin{block}{Slides}
\centering
openbiometerics.org/slides.pdf
\end{block}
\begin{block}{Source}
\centering
github.com/biometrics/openbr
\end{block}
\begin{block}{E-mail}
\centering
openbr-dev@googlegroups.com
\end{block}
\begin{center}
Thank You!
\end{center}
\end{frame}

\begin{frame}
\frametitle{Welcome to the Parallel Jungle!}
\begin{figure}
\includegraphics[width=\textwidth]{img/sutter}
\caption{http://www.drdobbs.com/parallel/welcome-to-the-parallel-jungle/232400273}
\end{figure}
\end{frame}

\begin{frame}[fragile]
\frametitle{Plugin Example: Local Binary Patterns}
{\footnotesize
\pause
\#include  \textless openbr\_plugin.h\textgreater \\
\pause
class LBP : public Transform \{ \\
\pause
\ \ BR\_PROPERTY(int, radius, 1) \\
\pause
\ \ void project(const Matrix \&src, Matrix \&dst) const \{ \\
\pause
\ \ \ \ for (int r=radius; r\textless src.rows-radius; r++) \\
\ \ \ \ \ \ for (int c=radius; c\textless src.cols-radius; c++) \{ \\
\ \ \ \ \ \ \ \ float cval = p[r*src.cols+c]; \\
\ \ \ \ \ \ \ \ dst(r, c) = \\
\ \ \ \ \ \ \ \ \ \ (p[(r-radius)*src.cols+c-radius] \textgreater = cval ? 128 : 0) \textbar \\
\ \ \ \ \ \ \ \ \ \ (p[(r-radius)*src.cols+c] \textgreater = cval ? 64 : 0) \textbar \\
\ \ \ \ \ \ \ \ \ \ ...; \\
 \ \ \ \ \ \ \} \\
\pause
\ \ \} \\
\}; \\
\pause
BR\_REGISTER(Transform, LBP) \\
\pause
... \\
Transform *lbp = Transform::make("LBP(1)"); \\
\pause
Transform *lbpu2 = Transform::make("LBP(1)+U2"); \\
}
\end{frame}

% End of slides
\end{document} 
